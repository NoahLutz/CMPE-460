% A skeleton file for producing Computer Engineering reports
% https://kgcoe-git.rit.edu/jgm6496/KGCOEReport_template

\documentclass[CMPE]{KGCOEReport}
\usepackage[colorinlistoftodos]{todonotes}
\usepackage{comment}
\usepackage{siunitx}

% Required packages
\usepackage{multicol}
\usepackage{multirow}

% The following should be changed to represent your personal information
\newcommand{\classCode}{CMPE-460}  % 4 char code with number
\newcommand{\name}{
Noah Lutz\\
Alejandro Vasquez-Lopez}
\newcommand{\LabSectionNum}{L2}
\newcommand{\LabInstructor}{Louis Beato}
\newcommand{\TAs}{
Nicholas Amatruda\\
\, Michael Baumgarten\\
\, John DeBrino
}
\newcommand{\LectureSectionNum}{1}
\newcommand{\LectureInstructor}{Louis Beato}
\newcommand{\exerciseNumber}{3}
\newcommand{\exerciseDescription}{Characterization of OPB745}
\newcommand{\dateDone}{9/13/19}
\newcommand{\dateSubmitted}{9/20/19}

\begin{document}
\maketitle
% Some reports require a table of contents.

\renewcommand\cftsecleader{\cftdotfill{\cftdotsep}} %Print dots between the sections and page numbers
\tableofcontents
\newpage
\setcounter{secnumdepth}{0} %Needed so that sections go into the TOC
\renewcommand\um{$\mu$m} %So you can pretty print micrometers
\newcommand\ua{$\mu$a }
\newcommand{\ol}[1]{\overline{#1}}

%%%%%%%%%%%%%%%%%%%%%%%%%%%%%%%%%%%%%%%%%%%%%%%%%%%%%
\section{Abstract}
The purpose of this exercise was to develop an understanding behind the 
functionality and various functional parameters of the OPB745 
photo-transducer. In order to test its ability to perform as an opto-isolator,
a tube with an adjustable reflective surface was created, and the OPB745 was placed inside in order to isolate any 
external light source. With the following implemented, the Opto-Isolator 
functioned accordingly yielding the expected results. 


%%%%%%%%%%%%%%%%%%%%%%%%%%%%%%%%%%%%%%%%%%%%%%%%%%%%%
\section{Design Methodology}
The first step in this lab exercise was to build the enclosure for the opto-isolator. 
Figure \ref{fig:enclosure} shows the logical diagram for the enclosure. 

\nimg[height=2.0in]{fig:enclosure}{fig/opto-isolator-enclosure.png}{Opto-Isolotor enclosure diagram}

Two black ABS plastic tubes were used to create this enclosure. One with roughly a 
one inch outer diameter with an end cap to hold the OPB754 sensor. The second 
tube, with a slightly smaller diameter, was used to hold the reflective surface
at a variable distance from the sensor as well as block out any interfering light from outside
the tube. A metric ruler was also attached to the inner tube to allow the distance from the
sensor to the reflective surface to be measured.\\

Once the enclosure was assembled the next step was to calculate the necessary 
resistors for the two different circuits used to analyze the voltage vs. distance
and frequency characteristics of the OPB745 sensor. Figures \ref{fig:circuit1} and 
\ref{fig:circuit2} show the two circuits. 

\nimg[height=2.5in]{fig:circuit1}{fig/circuit1.png}{Voltage vs. Distance circuit}

\nimg[height=1.5in]{fig:circuit2}{fig/circuit2.png}{Frequency response circuit}

Each circuit will be analyzed twice with two different load resistor values of 
$10\si{\kilo\ohm}$ and $20\si{\kilo\ohm}$. Equations \ref{equ:circuit1rf1} - \ref{equ:circuit1rf4}
show the derivation for the $R_f$ resistor for the circuit in Figure \ref{fig:circuit1}. The 
values for $V_f$ were obtained from the respective datasheets.

\begin{equation}
	\label{equ:circuit1rf1}
	I_f=\frac{V_{dc} - V_f}{R_f}
\end{equation}

\begin{equation}
	\label{equ:circuit1rf2}
	R_f=\frac{V_{dc} - V_f}{I_f}
\end{equation}

\begin{equation}
	\label{equ:circuit1rf3}
	R_f=\frac{5\si\volt - 1.7\si\volt}{40 \si{\milli\ampere}}
\end{equation}

\begin{equation}
	\label{equ:circuit1rf4}
	R_f= 82.5 \si\ohm
\end{equation}

The same process was used for deteminging the $R_f$ resistor for the circuit in Figure \ref{fig:circuit2}
and is shown in Equations \ref{equ:circuit2rf1} - \ref{equ:circuit2rf3}.

\begin{equation}
	\label{equ:circuit2rf1}
	R_f=\frac{V_{dc} - (V_{f,inverter} - V_{f,diode})}{I_f}
\end{equation}

\begin{equation}
	\label{equ:circuit2rf2}
	R_f=\frac{5 - (1.7\si\volt - 0.8\si\volt)}{40\si{\milli\ampere}}
\end{equation}

\begin{equation}
	\label{equ:circuit2rf3}
	R_f=65\si\ohm
\end{equation}


These circuits were then built using a breadboard, jumper wires and various 
resistors to get as close as possible to the calculated values above. The circuit
in Figure \ref{fig:circuit1} was analyzed by using a multimeter to measure the voltage 
at $V_{out}$ while the distance between the reflective surface and the sensor is varried.
The circuit in Figure \ref{fig:circuit2} was analyzed by using an oscilloscope at $V_{out}$ and 
a waveform generator at the input of the diode. The output was then measured while the frequency
of the input waveform was varried.



%%%%%%%%%%%%%%%%%%%%%%%%%%%%%%%%%%%%%%%%%%%%%%%%%%%%%
\section{Results \& Analysis}

Table \ref{tbl:circuit1} shows the measurements taken from the multimeter as the distance
from the OPB745 sensor and the reflective surface was varried.

\begin{table}[H]
	\caption{Voltage vs. Distance measurements}
	\label{tbl:circuit1}
	\centering
	\begin{tabular}{|c|c|c|c|c|}
		\hline
		& \multicolumn{2}{c|}{$R_{l,1}$ - $10\si{\kilo\ohm}$} & \multicolumn{2}{c|}{$R_{l,2}$ - $20\si{\kilo\ohm}$} \\ \hline
		Distance ($\si{\milli\meter}$) & $V_{out}$ ($\si\volt$)       & $I_{R_l}$ ($\si{\milli\ampere}$)      & $V_{out}$ ($\si\volt$)       & $I_{R_l}$ ($\si{\milli\ampere}$)      \\ \hline
		0             & 4.934          & 0.4934        & 4.859          & 0.24295       \\ \hline
		1             & 0.866          & 0.0866        & 0.781          & 0.03905       \\ \hline
		2             & 0.723          & 0.0723        & 0.679          & 0.03395       \\ \hline
		3             & 0.701          & 0.0701        & 0.653          & 0.03265       \\ \hline
		4             & 0.703          & 0.0703        & 0.651          & 0.03255       \\ \hline
		5             & 0.717          & 0.0717        & 0.662          & 0.0331        \\ \hline
		6             & 0.736          & 0.0736        & 0.681          & 0.03405       \\ \hline
		7             & 0.757          & 0.0757        & 0.707          & 0.03535       \\ \hline
		8             & 0.782          & 0.0782        & 0.729          & 0.03645       \\ \hline
		9             & 0.808          & 0.0808        & 0.729          & 0.03645       \\ \hline
		10            & 0.844          & 0.0844        & 0.772          & 0.0386        \\ \hline
		11            & 0.985          & 0.0985        & 0.791          & 0.03955       \\ \hline
		12            & 2.107          & 0.2107        & 0.809          & 0.04045       \\ \hline
		13            & 2.800          & 0.2800        & 0.838          & 0.0419        \\ \hline
		14            & 3.097          & 0.3097        & 0.885          & 0.04425       \\ \hline
		15            & 3.300          & 0.3300        & 1.137          & 0.05685       \\ \hline
		20            & 3.582          & 0.3582        & 1.613          & 0.08065       \\ \hline
		25            & 3.627          & 0.3627        & 1.985          & 0.09925       \\ \hline
		30            & 3.848          & 0.3848        & 2.246          & 0.1123        \\ \hline
		35            & 4.243          & 0.4243        & 3.255          & 0.16275       \\ \hline
		40            & 4.508          & 0.4508        & 3.971          & 0.19855       \\ \hline
		45            & 4.637          & 0.4637        & 4.287          & 0.2139        \\ \hline
		50            & 4.695          & 0.4695        & 4.421          & 0.22105       \\ \hline
	\end{tabular}
\end{table}

The data in Table \ref{tbl:circuit1} was graphed using Excel and produced Figures \ref{fig:circuit1_voltage}
and \ref{fig:circuit1_current}. These figures show data produced by the circuit with both the $10\si{\kilo\ohm}$
and $20\si{\kilo\ohm}$ load resistors. From inspection the voltage at the $0\si{\milli\meter}$ mark reads close to that
of the source, $5\si\volt$. As the distance initially increases, the voltage drops close to 0 and as the distance continues
to grow, the voltage begins to return to the source value of $5\si\volt$. This is due to the operating range of the device. 

\nimg[height=2.5in]{fig:circuit1_voltage}{fig/circuit1_voltage.png}{Voltage vs. Distance of OPB745}
\nimg[height=2.5in]{fig:circuit1_current}{fig/circuit1_current.png}{Current vs. Distance of OPB745}



Results and Analysis go here. Follow lab report guidelines. Remember to introduce Figures (\ref{img:dog}) through (\ref{img:lindbloom} before they appear in the document. Make sure that your figure labels are below the figure itself and that they have a title.


A pair of exemplary segmentations are illustrated in Figures (\ref{img:dog_CPU}) and (\ref{img:dog_GPU}), corresponding to the CPU and GPU implementations respectively. Note that the images are identical.


Graphs are very helpful if you're trying to show a trend to the reader. Figures (\ref{fig:time24}), (\ref{fig:time100}), and (\ref{fig:time400}) plot the CPU and GPU execution times of the segmentation of the images of different sizes, with 24, 100, and 400 superpixels respectively. Note that not only does the GPU consistently provide a speedup >1, but that the speedup improves with an increasing image size. Moreover, speedup also improves with an increased superpixel size.


Results and Analysis are also pretty tedious :)\\

%%%%%%%%%%%%%%%%%%%%%%%%%%%%%%%%%%%%%%%%%%%%%%%%%%%%%%%%%%%%%%%%%%%%%%%

\section{Conclusions}
Conclusions go here. Follow lab report guidelines.\\


%%%%%%%%%%%%%%%%%%%%%%%%%%%%%%%%%%%%%%%%%%%%%%%%%%%%%%%%%%%%%%%%%%%%%%%



\end{document}

