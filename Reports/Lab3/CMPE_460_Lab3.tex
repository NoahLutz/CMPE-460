% A skeleton file for producing Computer Engineering reports
% https://kgcoe-git.rit.edu/jgm6496/KGCOEReport_template

\documentclass[CMPE]{KGCOEReport}
\usepackage[colorinlistoftodos]{todonotes}
\usepackage{comment}
\usepackage{siunitx}

% Required packages
\usepackage{multicol}
\usepackage{multirow}
\usepackage{longtable}

% The following should be changed to represent your personal information
\newcommand{\classCode}{CMPE-460}  % 4 char code with number
\newcommand{\name}{
Noah Lutz\\
Alejandro Vasquez-Lopez}
\newcommand{\LabSectionNum}{L2}
\newcommand{\LabInstructor}{Louis Beato}
\newcommand{\TAs}{
Nicholas Amatruda\\
\, Michael Baumgarten\\
\, John DeBrino
}
\newcommand{\LectureSectionNum}{1}
\newcommand{\LectureInstructor}{Louis Beato}
\newcommand{\exerciseNumber}{3}
\newcommand{\exerciseDescription}{Characterization of OPB745}
\newcommand{\dateDone}{9/13/19}
\newcommand{\dateSubmitted}{9/20/19}

\begin{document}
\maketitle
% Some reports require a table of contents.

\renewcommand\cftsecleader{\cftdotfill{\cftdotsep}} %Print dots between the sections and page numbers
\tableofcontents
\newpage
\setcounter{secnumdepth}{0} %Needed so that sections go into the TOC
\renewcommand\um{$\mu$m} %So you can pretty print micrometers
\newcommand\ua{$\mu$a }
\newcommand{\ol}[1]{\overline{#1}}

%%%%%%%%%%%%%%%%%%%%%%%%%%%%%%%%%%%%%%%%%%%%%%%%%%%%%
\section{Abstract}
The purpose of this exercise was to develop an understanding behind the 
functionality and various functional parameters of the OPB745 
opto-isolator. In order to test its ability to perform as an opto-isolator,
a tube with an adjustable reflective surface was created, and the OPB745 was placed inside in order to isolate it from any 
external light source. The sensor was tested in two distinct ways. The first was the voltage and current characteristics
versus distance from the reflective surface. This was observed by measuring the voltage output across the sensor 
while loaded while the reflective surface was moved relative to the sensor. The second was the frequency capabilities 
of the sensor. This was performed using a waveform generator to generate waveforms of various frequencies and measuring 
the output using an oscilloscope. With the following implemented, the Opto-Isolator functioned accordingly yielding the expected results. 


%%%%%%%%%%%%%%%%%%%%%%%%%%%%%%%%%%%%%%%%%%%%%%%%%%%%%
\section{Design Methodology}
The first step in this lab exercise was to build the enclosure for the opto-isolator. 
Figure \ref{fig:enclosure} shows the logical diagram for the enclosure. 

\nimg[height=2.0in]{fig:enclosure}{fig/opto-isolator-enclosure.png}{Opto-Isolotor enclosure diagram}

Two black ABS plastic tubes were used to create this enclosure. One with roughly a 
one inch outer diameter with an end cap to hold the OPB754 sensor. The second 
tube, with a slightly smaller diameter, was used to hold the reflective surface
at a variable distance from the sensor as well as block out any interfering light from outside
the tube. A metric ruler was also attached to the inner tube to allow the distance from the
sensor to the reflective surface to be measured.\\

Once the enclosure was assembled the next step was to calculate the necessary 
resistors for the two different circuits used to analyze the voltage vs. distance
and frequency characteristics of the OPB745 sensor. Figures \ref{fig:circuit1} and 
\ref{fig:circuit2} show the two circuits. 

\nimg[height=2.5in]{fig:circuit1}{fig/circuit1.png}{Voltage vs. Distance circuit}

\nimg[height=1.5in]{fig:circuit2}{fig/circuit2.png}{Frequency response circuit}

Equations \ref{equ:circuit1rf1} - \ref{equ:circuit1rf4}
show the derivation for the $R_f$ resistor for the circuit in Figure \ref{fig:circuit1}. The 
values for $V_f$ were obtained from the respective datasheets.

\begin{equation}
	\label{equ:circuit1rf1}
	I_f=\frac{V_{dc} - V_f}{R_f}
\end{equation}

\begin{equation}
	\label{equ:circuit1rf2}
	R_f=\frac{V_{dc} - V_f}{I_f}
\end{equation}

\begin{equation}
	\label{equ:circuit1rf3}
	R_f=\frac{5\si\volt - 1.7\si\volt}{40 \si{\milli\ampere}}
\end{equation}

\begin{equation}
	\label{equ:circuit1rf4}
	R_f= 82.5 \si\ohm
\end{equation}

The same process was used for deteminging the $R_f$ resistor for the circuit in Figure \ref{fig:circuit2}
and is shown in Equations \ref{equ:circuit2rf1} - \ref{equ:circuit2rf3}. Again, the values for $V_f$ were
obtained from the appropriate data sheets.

\begin{equation}
	\label{equ:circuit2rf1}
	R_f=\frac{V_{dc} - (V_{f,inverter} - V_{f,diode})}{I_f}
\end{equation}

\begin{equation}
	\label{equ:circuit2rf2}
	R_f=\frac{5 - (1.7\si\volt - 0.8\si\volt)}{40\si{\milli\ampere}}
\end{equation}

\begin{equation}
	\label{equ:circuit2rf3}
	R_f=65\si\ohm
\end{equation}


These circuits were then built using a breadboard, jumper wires and various 
resistors to get as close as possible to the calculated values above.
Each circuit was analyzed twice with two different load resistor values of 
$10\si{\kilo\ohm}$ and $20\si{\kilo\ohm}$. \\

The circuit in Figure \ref{fig:circuit1} was analyzed by using a multimeter to measure the voltage 
at $V_{out}$ while the distance between the reflective surface and the sensor is varried.
This gives the voltage vs. distance characteristics of the OPB745 sensor.\\

The circuit in Figure \ref{fig:circuit2} was analyzed by using an oscilloscope at $V_{out}$ and 
a waveform generator at the input of the diode. The output was then measured while the frequency
of the input waveform was varried starting at $100\si\hertz$ and increasing by $100\si\hertz$ steps
until the output of the OPB745 sensor no longer resembled the input waveform. A square wave with a duty
cyle of $50\%$ was used for this lab exercise. This shows the frequency limitations of the OPB745, specifically
the frequency limitations of the photo-transistor.



%%%%%%%%%%%%%%%%%%%%%%%%%%%%%%%%%%%%%%%%%%%%%%%%%%%%%
\section{Results \& Analysis}

Table \ref{tbl:circuit1} shows the measurements taken from the multimeter as the distance
from the OPB745 sensor and the reflective surface was varried. Measurments were taken using the
circuit in Figure \ref{fig:circuit1}.

\begin{longtable}[c]{|c|c|c|c|c|}
\caption{Voltage/Current vs. Distance measurements}
\label{tbl:circuit1}
\hline
		& \multicolumn{2}{c|}{$R_{l,1}$ - $10\si{\kilo\ohm}$} & \multicolumn{2}{c|}{$R_{l,2}$ - $20\si{\kilo\ohm}$} \\ \hline
\endfirsthead
%
\endhead
%
		Distance ($\si{\milli\meter}$) & $V_{out}$ ($\si\volt$)       & $I_{R_l}$ ($\si{\milli\ampere}$)      & $V_{out}$ ($\si\volt$)       & $I_{R_l}$ ($\si{\milli\ampere}$)      \\ \hline
0             & 4.934          & 0.4934        & 4.859          & 0.24295       \\ \hline
1             & 0.866          & 0.0866        & 0.781          & 0.03905       \\ \hline
2             & 0.723          & 0.0723        & 0.679          & 0.03395       \\ \hline
3             & 0.701          & 0.0701        & 0.653          & 0.03265       \\ \hline
4             & 0.703          & 0.0703        & 0.651          & 0.03255       \\ \hline
5             & 0.717          & 0.0717        & 0.662          & 0.0331        \\ \hline
6             & 0.736          & 0.0736        & 0.681          & 0.03405       \\ \hline
7             & 0.757          & 0.0757        & 0.707          & 0.03535       \\ \hline
8             & 0.782          & 0.0782        & 0.729          & 0.03645       \\ \hline
9             & 0.808          & 0.0808        & 0.729          & 0.03645       \\ \hline
10            & 0.844          & 0.0844        & 0.772          & 0.0386        \\ \hline
11            & 0.985          & 0.0985        & 0.791          & 0.03955       \\ \hline
12            & 2.107          & 0.2107        & 0.809          & 0.04045       \\ \hline
13            & 2.800          & 0.2800        & 0.838          & 0.0419        \\ \hline
14            & 3.097          & 0.3097        & 0.885          & 0.04425       \\ \hline
15            & 3.300          & 0.3300        & 1.137          & 0.05685       \\ \hline
20            & 3.582          & 0.3582        & 1.613          & 0.08065       \\ \hline
25            & 3.627          & 0.3627        & 1.985          & 0.09925       \\ \hline
30            & 3.848          & 0.3848        & 2.246          & 0.1123        \\ \hline
35            & 4.243          & 0.4243        & 3.255          & 0.16275       \\ \hline
40            & 4.508          & 0.4508        & 3.971          & 0.19855       \\ \hline
45            & 4.637          & 0.4637        & 4.287          & 0.2139        \\ \hline
50            & 4.695          & 0.4695        & 4.421          & 0.22105       \\ \hline
\end{longtable}

The data in Table \ref{tbl:circuit1} was graphed using Excel and produced Figures \ref{fig:circuit1_voltage}
and \ref{fig:circuit1_current}. These figures show data produced by the circuit with both the $10\si{\kilo\ohm}$
and $20\si{\kilo\ohm}$ load resistors. From inspection the voltage at the $0\si{\milli\meter}$ mark reads close to that
of the source, $5\si\volt$. As the distance initially increases, the voltage drops close to 0 and as the distance continues
to grow, the voltage begins to return to the source value of $5\si\volt$. This is due to the operating range of the device. 

\nimg[height=2.5in]{fig:circuit1_voltage}{fig/circuit1_voltage.png}{Voltage vs. Distance of OPB745}
\nimg[height=2.5in]{fig:circuit1_current}{fig/circuit1_current.png}{Current vs. Distance of OPB745}

The next part of the lab exercise focused on the frequency limitations of the OPB745 sensor. This was done using the circuit in
Figure \ref{fig:circuit2}, along with a waveform generator to generate input waveforms for the IR LED embedded in the 
OPB745 and an oscilloscope to measure the output of the photo-transistor. \\

Figures \ref{fig:circuit2_10k_100Hz} and \ref{fig:circuit2_20k_100Hz} show the oscilloscope captures for
the circuit in Figure \ref{fig:circuit2} for both the $10\si{\kilo\ohm}$ and $20\si{\kilo\ohm}$ load resistors
with an input waveform of a $100\si\hertz$ and a $50\%$ duty cycle. The top waveform shows the input waveform and
the bottom waveform is the output waveform.

\nimg[height=2.5in]{fig:circuit2_10k_100Hz}{fig/scope_circuit2_10k_100Hz.png}{Oscilloscope capture with $10\si{\kilo\ohm}$ load at $100\si\hertz$}
\nimg[height=2.5in]{fig:circuit2_20k_100Hz}{fig/scope_circuit2_20k_100Hz.png}{Oscilloscope capture with $20\si{\kilo\ohm}$ load at $100\si\hertz$}

Figures \ref{fig:circuit2_10k_1kHz} and \ref{fig:circuit2_20k_600Hz} show the oscilloscope captures where the 
OPB745 is no longer able to handle the input waveform frequency. Instead of seeing a nice square wave output, 
all that is seen is a steady signal. The point at which this happens is different depending on the load resistor.
For the $10\si{\kilo\ohm}$ resistor the OPB745 sensor stopped working at around $1\si{\kilo\hertz}$ while for the
$20\si{\kilo\ohm}$ resistor, the breakdown point was only around $600\si\hertz$. Again, the top signal is the input
waveform and the bottom signal is the output waveform.

\nimg[height=2.5in]{fig:circuit2_10k_1kHz}{fig/scope_circuit2_10k_1kHz.png}{Oscilloscope capture with $10\si{\kilo\ohm}$ load at $1\si{\kilo\hertz}$}
\nimg[height=2.5in]{fig:circuit2_20k_600Hz}{fig/scope_circuit2_20k_600Hz.png}{Oscilloscope capture with $20\si{\kilo\ohm}$ load at $600\si{\hertz}$}

\section{Questions}
\begin{enumerate}
	\item Schmitt inverter reduces noise and produces a more defined square wave when taking an analog signal (opto-isolator) and converting it to a digital signal (oscope). The difference between the two devices are that the 74LS14 with Schmitt trigger will sustain the output value until the input has an adequate amount of change to trigger a change in the output.
	\item The voltage initially starts off at 5V at 0mm because there is no change in the opto-isolators positioning. As the distance is increased, the voltage drops to close to 0V, the reasoning behind this is due to the operating range of the OPB745. Based on the data sheet, the operating range is from 0 inches to 0.5 inches which is the distance to the reflective surface. At 0mm, the device is out of its range and no IR light from the LED can reach the photo-transistor, meaning no current is flowing through the transistor and therefore, $V_{out} = 5V$.
	\item The frequency changes when replacing the 10k load resistor with the 20k resistor due to the rise and fall times of the OPB745. The OPB745 rise and fall times increase until it reaches a 1k load resistance, any increase in load resistance from this point on will cause the frequency to drop, therefore the decrease was anticipated. 
\end{enumerate}
%%%%%%%%%%%%%%%%%%%%%%%%%%%%%%%%%%%%%%%%%%%%%%%%%%%%%%%%%%%%%%%%%%%%%%%

\section{Conclusions}
Opto-isolators are very common devices found in many fields of interest. There are applications 
where physical measurements are not feasible. Opto-isolators are the solution to these issues as 
they provide a means of measurement that is isolated from the actual unit. Understanding the fundamentals 
of how one operates is imperative to being able to create well-designed systems. The exercise proved to be an overall success.


%%%%%%%%%%%%%%%%%%%%%%%%%%%%%%%%%%%%%%%%%%%%%%%%%%%%%%%%%%%%%%%%%%%%%%%



\end{document}

